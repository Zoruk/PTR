% Compiler ce document 

% package de base
\documentclass[10pt,a4paper]{article}
\usepackage[utf8]{inputenc}

% langues
\usepackage[francais]{babel}
\usepackage[french]{datetime}
\usepackage[T1]{fontenc}
\usepackage{amsmath}
\usepackage{amsfonts}
\usepackage{amssymb}
\usepackage{graphicx}

% liens
\usepackage[hidelinks]{hyperref}

% Page de Garde -- Necessite d'installer le package titling, si probleme
% commenter la ligne suivante ainsi que les infos necessaires a la page
% de garde
\usepackage{pageGarde/HEIG_STY}

% En tetes / bas de page
\usepackage{fancyhdr}
\pagestyle{fancy}
\fancyhfoffset[re]{4cm}

% Pour le listing de code
\usepackage{listings}

%Pour le landscape
\usepackage{lscape}

%marge des pages
\setlength{\textwidth}{16cm}
\setlength{\textheight}{24cm}
\setlength{\oddsidemargin}{0cm}
\setlength{\voffset}{-1.5cm}
\setlength{\headheight}{15pt}

%% Sans-serif Arial-like fonts
\renewcommand{\rmdefault}{phv} 
\renewcommand{\sfdefault}{phv} 
\usepackage{tabularx}
\usepackage{graphicx}
\usepackage{eurosym}
\usepackage{xspace}
\newcommand{\projectname}[0]{LTANR\xspace} 
\newcommand{\tab}[1]{\hspace{.2\textwidth}\rlap{#1}}

%No indent for the first line
\setlength{\parindent}{0cm}

% IFor code include
\usepackage{listings}
\usepackage{xcolor}
\usepackage{xargs}
\usepackage{xifthen}

\definecolor{mygreen}{rgb}{0,0.6,0}
\definecolor{mygray}{rgb}{0.5,0.5,0.5}
\definecolor{mymauve}{rgb}{0.58,0,0.82}

\lstdefinestyle{customc}{
  belowcaptionskip=1\baselineskip,
  breaklines=true,
  frame=L,
  xleftmargin=\parindent,
  language=C,
  numbers=left,
  numbersep=8pt,
  stepnumber=1,
  numberstyle=\tiny\color{gray},
  showstringspaces=false,
  tabsize=4,
  basicstyle=\footnotesize\ttfamily,
  keywordstyle=\bfseries\color{green!40!black},
  commentstyle=\itshape\color{purple!40!black},
  identifierstyle=\color{blue},
  stringstyle=\color{orange},
}

\lstdefinestyle{customcpp}{
   belowcaptionskip=1\baselineskip,
   breaklines=true,
   frame=L,
   xleftmargin=\parindent,
   language=C++,
   showstringspaces=false,
   basicstyle=\footnotesize\ttfamily,
   keywordstyle=\color{blue},
   commentstyle=\itshape\color{green},
   stringstyle=\color{purple},
   identifierstyle=\color{gray},
   tabsize=4,
   numbers=left,
   numbersep=8pt,
   stepnumber=1,
   numberstyle=\tiny\color{gray},
}

\newcommandx{\includecode}[3][1=, 2=c]{
	\ifthenelse{\equal{#1}{}}
	    {\lstinputlisting[caption=#3, escapechar=, style=custom#2]{#3}}
	    {\lstinputlisting[caption=#1, escapechar=, style=custom#2]{#3}}
}



%parametters
\author{Loïc Haas}
\title{Laboratoire 03}

% Informations necessaires a la page de garde
% Commenter si probleme de compilation
\acro{PTR}
%\prof{Yann Thoma}
\date{\today}

%en-tête
\lhead{}
\chead{}
\rhead{}

%pied-de-page
\lfoot{\today}
\cfoot{HEIG-VD}
\rfoot{page \thepage}


\begin{document}
\maketitle
\newpage
\tableofcontents
\newpage

\section{Introduction}
Le but principal du laboratoire est de se familiariser avec les taches Xenomai et de comparer la précision des tache periodiques entre \textsc{Pthread} et \textsc{Xenomai} lorsque le system est surchargé. Toutes les mesures effectuée dans ce laboratoire sont sur un seul cœur CPU avec l'application \textit{dohell} exécutée simultanément.

\section{Résultats}
\subsection{Pthread}
Voici les résultat de l'intervalle entre deux exécutions de la tache periodique que l'on obtiens avec le code page \pageref{codePthread} pour 10000 mesures avec un intervalle d'une milliseconde les valeurs sont donné en nanosecondes.\\
\begin{verbatim}
  Total of 10000 values 
    Minimum  = 1005730.000000 (position = 6750) 
    Maximum  = 383418316.000000 (position = 2049) 
    Sum      = 10890408840.000000 
    Mean     = 1089040.884000
    Variance = 14619504619410.791016 
    Std Dev  = 3823546.079154 
    CoV      = 3.510930
\end{verbatim}

\subsection{Xenomai}
Voici les résultat de l'intervalle entre deux exécutions de la tache periodique que l'on obtiens avec le code page \pageref{codeXenomai} pour 10000 mesures avec un intervalle d'une milliseconde les valeurs sont donné en nanosecondes.\\
\begin{verbatim}
  Total of 10000 values 
    Minimum  = 992747.000000 (position = 5609) 
    Maximum  = 1007146.000000 (position = 5608) 
    Sum      = 9999999385.000000 
    Mean     = 999999.938500 
    Variance = 247435.016846 
    Std Dev  = 497.428404 
    CoV      = 0.000497
\end{verbatim}
\subsection{Comparaison des résultats}
On vois bien que l’écart type d'une tache \textsc{Xenomai} ($0.497\mu s$) est bien plus petit que celui d'une tache \textsc{Pthread} ($3.824ms$) avec un rapport de 7000 on se rend bien compte que les taches \textsc{Xenomai} ne sont presques pas perturbée par les application s’exécutant dans le user space.

\newpage
\section{Code}
\subsection{Pthread}
\label{codePthread}
\begin{scriptsize}
\includecode[\detokenize{pthread_timer.c}]{../Sources/pthread_timer/pthread_timer.c}
\end{scriptsize}

%\newpage
\subsection{Xenomai}
\label{codeXenomai}
\begin{scriptsize}
\includecode[\detokenize{xenomai_timer.c}]{../Sources/xenomai_timer/xenomai_timer.c}
\end{scriptsize}

\end{document}